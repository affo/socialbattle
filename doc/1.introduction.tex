\chapter{Introduction}
	Before talking about the project in its details, I think it is worth to spend a few words about the general 
	requirements imposed by the nature of the Service Technologies Course from which the project comes from.\\
	We have also to keep in mind that this project will be used to test the performance of a IaaS system.

	\noindent Given this, I will have to fulfill three specific requirements:
	\begin{itemize}
		\item The system will have to expose some documented services through an API;
		\item The system has to integrate other services;
		\item The system has to be computationally intensive, both for the database and the CPU.
	\end{itemize}
	\section{Project Description}
		The project aims at mixing the features of a social network and a role-playing game 
		into a web application named \textit{Social Battle}.

		\noindent To better define the context we specify the main features of the two elements:
		\begin{itemize}
			\item \textbf{Social network}\\
			The main features of a social network are
			the relationship with other users and 
			the interaction with them.\\
			The relation could be a friendship (a symmetric one) or a fellowship (an asymmetric one).\\
			The interaction among (or between) users could be chosen among posts, instant messaging, 
			deferred messaging, votes, comments, tags and others.
			\item \textbf{Role-playing game}\\
			The central idea of a role-playing game is the growth of the character (or the team) we 
			are using, in terms of the level of experience and the equipments and objects in general.\\
			There are a lot of RPGs (especially the one played exclusively on-line) in which there is 
			no other goal than skilling more and more your main character.
		\end{itemize}

		\noindent \textit{Social Battle} will take most of the features of a social network adding the 
		ones of a role-playing game.

		\noindent Thinking think of \textit{Facebook} as our landmark for the social networking part, 
		you can find below the matching between some \textit{Facebook} components and what 
		they might become in \textit{Social Battle} context:
		\begin{itemize}
			\item \textbf{The page}\\
			Something as a location. A place in which a user can fight mobs.

			\item \textbf{The friendship}\\
			It could become an asymmetric relationship, suck as \textit{Twitter}'s fellowship.
			Once \textit{User A} follows \textit{User B}, he could follow all the progress
			of \textit{User B};

			\item \textbf{The post}\\
			Could replace loots.\\
			The \textit{like} on a post could become the grabbing of a loot.

			\item \textbf{Instant messaging}\\
			The fight between a user and a mob (PVE) or between two users (PVP).
		\end{itemize}

		\noindent These were only examples to better understand what \textit{Social Battle} could become.\\
		As you may have already understood, the system will not be provided with a GUI except from the one 
		that usually a social network gives, and so a textual one.

	\section{Glossary}
	I think it is worth to define some common words that will be used in this documentation.
	\begin{itemize}
		\item \textbf{SB}: \textit{Social Battle}, the system we are talking about.
		\item \textbf{RPG}: Role-playing game.
		\item \textbf{Mob}: An enemy controlled by an artificial intelligence (maybe the server).\\
			The user will have to fight against the mob to obtain experience (money, objects, equipment)
			and to skill his character.
		\item \textbf{Spawn}: The act of ``appearing'' of a mob.\\
			The idea is that, if the user is at a location (a web page in our case), mobs will appear 
			randomly and the user will have to fight them.
		\item \textbf{Guil}: Term borrowed from \textit{Final Fantasy}.\\
			A guil is the virtual currency of the RPG. Generically ``money''.
		\item \textbf{Loot}: The items dropped by a mob when it has been defeated by the user.
		\item \textbf{EXP}: The experience points earned by the user after defeating a mob.
		\item \textbf{Equip}: Abbreviation of ``equipment''.
	\end{itemize}

	\section{Goals}
	To identify the goals, we have to think of what our system has to provide both to the end user 
	(in terms of a web application) and both to the developer (in terms of the API).\\
	My aim is to develop both the API and a web application that consumes them.\\
	For now, when I talk about goals, I mean the goals of the application that the end user will use.

	\noindent So, the final system should give the user the possibility to, at least:
	\newcommand{\goalSignup}{Sign up (possibly using a pre-existent social network) and sign in;}
	\newcommand{\goalSkill}{Skill his character and so fight against mobs (maybe other users) to gain
			EXP, equipments, loots and guils.}
	\newcommand{\goalFollow}{Follow some characters to learn about their progress and be followed by other users in turn.}
	\newcommand{\goalProfile}{Customize his character, at least in terms of the items equipped.}
	\newcommand{\goalExplore}{Explore the world around him to meet new players and to fight against different mobs.}
	\newcommand{\goalShare}{Share his results (about battles, earned badges or others) on a social network.}
	\begin{itemize}
		\item \goalSignup
		\item \goalSkill
		\item \goalFollow
		\item \goalProfile
		\item \goalExplore
		\item \goalShare
	\end{itemize}

	\noindent It is obvious that the API should be thought with the purpose of making it possible to the developer 
	(which in this case it is me) to build a web application that fulfills the goals explained above.\\
	In later chapters a detailed explanation of the API will be provided.

	\section{Actors}
	The Actors in our system are basically
	\begin{itemize}
		\item \textbf{The guest}\\
		The unregistered user which can only sign up to \textit{Social Battle}.
		\item \textbf{The user}\\
		The registered user, which is only a guest which has signed up.
		\item \textbf{The mob}\\
		It is only an artificial intelligence spawned by the server itself, but it can be included as an actor in 
		our system. Even if it is not handled by a human mind it is the subject around which \textit{Social Battle} 
		will be built.
	\end{itemize}
	The topic about which action an actor can perform in the system will be deepened in the \textit{Use Case} section.