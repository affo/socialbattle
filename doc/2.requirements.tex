\chapter{Requirements and Specifications}
	In this chapter the requirements for our web application (the one the end user will use) 
	and for the APIs (that a possible developer will use) will be listed.

	It is important to say that the requirements that will be listed below, are the 
	theoretical requirements of a \textit{full} application.\\
	I will write a paragraph in which I will specify the minimum functionalities of a working application.

	So, from the goals written in the previous chapter we can derive our general requirements.\\
	I will write down here, for each goal the requirements that will fulfill it.
	\begin{itemize}
		\item \textbf{\goalSignup}
			\begin{itemize}
				\item Provide a sign up functionality (at least with \Facebook{}).
				\item Provide a sign up functionality not using an already existent social network.
				\item Provide a sign in functionality.
			\end{itemize}
		\item \textbf{\goalExplore}
			\begin{itemize}
				\item Provide something like a ``location'' in which a player can simply chat with 
				other players and find mobs.
			\end{itemize}
		\item \textbf{\goalSkill}
			\begin{itemize}
				\item Provide the possibility to fight against a spawned mob according to the abilities and 
				skills of the character.\\
				This means that the server shouldn't spawn mobs that have a level too high compared to the one 
				of the player.
				\item Every time a player defeats an enemy the system should reward the player giving him 
				EXP, guils and so on.
				\item There should exist a place in which the user can manage the status of his character in terms of 
				the equipment, the abilities in use and the items collected.
				\item Provide a PVP functionality.
			\end{itemize}
		\item \textbf{\goalFollow}
			\begin{itemize}
				\item Provide the possibility to a user to follow another one.
				\item If \textit{User A} follows \textit{User B}, the system should properly notify 
				\textit{User A} about \textit{User B}'s progresses.
			\end{itemize}
		\item \textbf{\goalProfile}
			\begin{itemize}
				\item Provide a ``edit profile'' functionality in terms of the avatar and other personal 
				information. Not about character's skills.
			\end{itemize}
		\item \textbf{\goalSocial}
			\begin{itemize}
				\item Let the user post and comment. This functionalities will have to be well integrated with the aim of the system. The post and the comment will be an extra to skill the character.
			\end{itemize}
		\item \textbf{\goalShare}
			\begin{itemize}
				\item Provide the possibility to share (using the social network which the user signed in with) his 
				progresses (defeating a mob, earning guils, earn equips or earn badges).
			\end{itemize}
	\end{itemize}

	\section{Functional Requirements}
		From the general requirements wrote above we can define the functional requirements 
		and so the functionalities that the system has to provide to users. A guest can only sign up to the system; 
		a user, instead, can exploit the entire set of features provides, so:
		\begin{itemize}
			\item Sign in;
			\item Modify his profile;
			\item Explore locations;
			\item Change equipments;
			\item Add/remove abilities;
			\item Manage items;
			\item Follow/stop following another player;
			\item Share his results;
			\item Fight against a mob;
			\item Fight against a player;
			\item Post;
			\item Comment;
			\item Sign out.
		\end{itemize}
	\section{Non Functional Requirements: the API}
		First of all the APIs to \textit{GET} and \textit{POST} data from/to the system will be necessary. So, in general, 
		services as the sign up, posting, comment, \textit{like} and other social facilities will be provided.\\
		The same thing stands for the peculiarities of the platform, such as, activate an ability, use an object, collect an item and so on. In general, all the possibilities to modify the persistent data will be exposed.\\
		To built this part of the back-end of the application, I will get ideas from \textit{Facebook's Graph API} which I think it is a great model to build a well organized exposed API.

		In addition to this general services, I think it is worth to provide a \textit{Battle Service}, to let any system 
		include something like a \textit{widget} that will include a full \textit{SB} battle. In this way, any system will give the possibility to a \SocialBattle{} user, to face random mobs even outside \textit{SB} itself.\\
		It would be great if we also give the possibility to fight against a custom mob (in addition of the widget). Thanks to this anyone who includes the \textit{Battle Service} will be able to advertise his brand/company through mobs and enrich the game itself. For example \Facebook{} could think of the ``Mark Zuckerberg'' custom mob; \textit{Coca Cola} the ``Polar Bear'' mob; \textit{Stack Overflow} the ``Bad-Code'' mob and so on.\\

		Further considerations about the APIs will be made in Chapter 5.
	\section{Specifications}
		Here are some further specification to make better understand how the system will work:
		\begin{itemize}
			\item The ``location'' should be something like a \Facebook{} page, a ``Room'' with its own features. For instance in a specific room a user can find some mobs and some others not.
			\item Once a user is in a room, mobs will be spawned randomly.
			\item The battle will be an ATB (Active Time Battle) one. So, any attack (ability) to be performed will require a certain specific time to be performed. Once the user selects an ability, he will have to wait for the time required. After that time passes, the ability will be performed and the user will be able to choose another ability; the battle goes on following this steps.
			\item The EXP and guils gained from a battle have to be proportional to the power of the mob.
			\item When mobs are defeated, they will leave items (loots, objects and/or equipment) and guils to the winner.
			\item If a user looses a battle (i.e. he dies), it will not be ``game over''. There will be a penalty because of the KO. The penalty could be applied to guils or to the EXP.
			\item Some rooms in which a user cannot fight mobs will be provided. These rooms (we will call them ``relax rooms'') will be locations in which a user can chat with some artificial intelligence to buy items; post freely to interact with other users and, maybe, exchange items with them.\\
			There would be also PVP locations (``Arenas'') in which a user can start fighting (through a chat mechanism) against another user.
		\end{itemize}

	\section{Basic functionalities}
	I think that it will be very difficult to develop the system described above in few months. So I will shrink the features in some basic one. I will list down here the functionalities which will be realized and which will give a minimal working application:
	\begin{itemize}
		\item Sign up and sign in;
	 	\item The battle (against mobs), which is the core of the application;
	 	\item The spawning of mobs;
	 	\item The post and the comments;
	 	\item The following mechanism;
	 	\item The rooms;
	 	\item The possibility to skill the character (at least equipment and abilities).
	\end{itemize} 