\chapter{UML Diagrams}
	\section{Use Case Diagram}
	The Use Case Diagram is needed to better which actions any actor can perform in the  system.\\
	For a better comprehension of the diagrams we will separate the diagrams per actor and per functionality (in the case of the registered user).

		\subsection{The Guest}
		The guest use case diagram is simple, because he can only register to the system:
		\begin{figure}[ht!]
			\centering
			\includegraphics[width=0.9\textwidth]{images/use_case_guest}
			\caption{Guest Use Case Diagram}
			\label{fig:use_case_guest}
		\end{figure}

		\subsection{The User}
		In the case of the user, things become more complex. Separating the different set of actions will help in understanding the use case diagram:
		\begin{itemize}
		 	\item \textbf{Sign in}
			\begin{figure}[ht!]
				\centering
				\includegraphics[width=0.9\textwidth]{images/use_case_user_sign_in}
				\caption{User sign in and sign out Use Case Diagram}
				\label{fig:use_case_user_sign_in}
			\end{figure}

			\item \textbf{Social \ldots} \hfill \\
			Any use case pointed by the actor \textit{includes} the use case ``Sign in''.
			\begin{figure}[ht!]
				\centering
				\includegraphics[width=0.9\textwidth]{images/use_case_user_social}
				\caption{Use Case Diagram for the social features}
				\label{fig:use_case_user_social}
			\end{figure}

			\item \textbf{\ldots Battle} \hfill \\
			Any use case pointed by the actor \textit{includes} the use case ``Sign in''.
			\begin{figure}[ht!]
				\centering
				\includegraphics[width=0.9\textwidth]{images/use_case_user_battle}
				\caption{Use Case Diagram for the ``battle'' features}
				\label{fig:use_case_user_battle}
			\end{figure}
		\end{itemize} 

	\section{Class Diagram}