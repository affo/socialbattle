\chapter{UML Diagrams}
	\section{Use Case Diagram}
	The Use Case Diagram is needed to better understand which actions any actor can perform in the  system and the dependencies between them.\\
	For a better comprehension of the diagrams we will separate the diagrams per actor and per functionality (in the case of the registered user).
\newpage
		\subsection{The Guest}
		The guest use case diagram is simple, because he can only register to the system:

		\begin{figure}[H]
			\centering
			\includegraphics[width=0.9\textwidth]{images/use_case_guest}
			\caption{Guest Use Case Diagram}
			\label{fig:use_case_guest}
		\end{figure}
\newpage
		\subsection{The User}
		In the case of the user, things become more complex. Separating the different set of actions will help in understanding the use case diagram:
		\begin{itemize}
		 	\item \textbf{Sign in}
			\begin{figure}[H]
				\centering
				\includegraphics[width=0.9\textwidth]{images/use_case_user_sign_in}
				\caption{User sign in and sign out Use Case Diagram}
				\label{fig:use_case_user_sign_in}
			\end{figure}
\newpage
			\item \textbf{Social \ldots} \hfill \\
			Any use case pointed by the actor \textit{includes} the use case ``Sign in''.
			\begin{figure}[H]
				\centering
				\includegraphics[width=0.9\textwidth]{images/use_case_user_social}
				\caption{Use Case Diagram for the social features}
				\label{fig:use_case_user_social}
			\end{figure}
\newpage
			\item \textbf{\ldots Battle} \hfill \\
			Any use case pointed by the actor \textit{includes} the use case ``Sign in''.
			\begin{figure}[H]
				\centering
				\includegraphics[width=0.9\textwidth]{images/use_case_user_battle}
				\caption{Use Case Diagram for the ``battle'' features}
				\label{fig:use_case_user_battle}
			\end{figure}
		\end{itemize} 
\newpage
	\section{Class Diagram}
	Eventually, to better understand the entities involved in our system and the interactions between them, we draw a simple class diagram:
	\begin{figure}[H]
		\centering
		\includegraphics[width=0.9\textwidth]{images/class_diagram}
		\caption{Class Diagram}
		\label{fig:class_diagram}
	\end{figure}
\newpage
	Even though the class diagram is complete, it cannot clarify the correspondence between the action that a user can perform and the room in which he is able to perform them.\\
	Even though this can be acknowledged analyzing carefully the use case diagrams, I think it would be useful to provide a state diagram relating rooms and the action that a user can perform in each of them:
	\begin{figure}[H]
		\centering
		\includegraphics[width=0.9\textwidth]{images/state_diagram}
		\caption{State Diagram for Rooms}
		\label{fig:state_diagram}
	\end{figure}